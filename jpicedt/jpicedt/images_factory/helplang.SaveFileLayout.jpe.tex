% Version: $Id: helplang.SaveFileLayout.jpe.tex,v 1.3 2013/02/28 21:46:44 vincentb1 Exp $
\ifx\JPicIsIncluded\undefined
\documentclass{article}
\usepackage[tightpage,active,psfixbb]{preview}
\usepackage[latin1]{inputenc}
\usepackage[svgnames]{xcolor}
\usepackage{array}
\usepackage{pstricks}
\providecommand*\LangId{fr}
\newcommand*\defFile{}
\edef\defFile{help-\LangId.SaveFileLayout.def}
\expandafter\input\expandafter{\defFile}
\begin{document}
\pagestyle{empty}
\begin{preview}
\fi
%%Created by jPicEdt 1.6-pre1: mixed JPIC-XML/LaTeX format
%%Mon Feb 25 21:46:51 CET 2013
%%Begin JPIC-XML
%<?xml version="1.0" standalone="yes"?>
%<jpic x-min="0" x-max="175" y-min="0" y-max="118" auto-bounding="false">
%<parallelogram fill-style="solid"
%	 stroke-style="none"
%	 p3="(120,65.3)"
%	 p2="(120,91.3)"
%	 fill-color="#afeeee"
%	 p1="(0,91.3)"
%	 />
%<parallelogram fill-style="solid"
%	 stroke-style="none"
%	 p3="(120,20.3)"
%	 p2="(120,62.8)"
%	 fill-color="#90ee90"
%	 p1="(0,62.8)"
%	 />
%<parallelogram fill-style="solid"
%	 stroke-style="none"
%	 p3="(120,0)"
%	 p2="(120,17.8)"
%	 fill-color="#dcdcdc"
%	 p1="(0,17.8)"
%	 />
%<parallelogram fill-style="solid"
%	 stroke-style="none"
%	 p3="(120,98)"
%	 p2="(120,118)"
%	 fill-color="#dcdcdc"
%	 p1="(0,118)"
%	 />
%<parallelogram fill-style="solid"
%	 stroke-style="none"
%	 p3="(120,91.3)"
%	 p2="(120,98)"
%	 fill-color="#ffb6c1"
%	 p1="(0,98)"
%	 />
%<parallelogram fill-style="solid"
%	 stroke-style="none"
%	 p3="(120,62.8)"
%	 p2="(120,65.3)"
%	 fill-color="#ffb6c1"
%	 p1="(0,65.3)"
%	 />
%<parallelogram fill-style="solid"
%	 stroke-style="none"
%	 p3="(120,17.8)"
%	 p2="(120,20.3)"
%	 fill-color="#ffb6c1"
%	 p1="(0,20.3)"
%	 />
%<text text-vert-align="top"
%	 text-icon="iconmode"
%	 anchor-point="(0.5,62.5)"
%	 text-hor-align="left"
%	 >
%\color{Blue}\tiny\begin{tabular}{@{}l@{}}
%\color{Brown}\%PSTricks content-type (pstricks.sty package needed)\\
%\color{Brown}\%Add \textbackslash usepackage\{pstricks\} in the preamble of your LaTeX file\\
%\color{Brown}\%You can rescale the whole picture (to 80\% for instance) by using the command \textbackslash def\textbackslash JPicScale\{0.8\}\\
%\textbackslash ifx\textbackslash JPicScale\textbackslash undefined\textbackslash def\textbackslash JPicScale\{1\}\textbackslash fi\\
%\textbackslash psset\{unit=\textbackslash JPicScale mm\}\\
%\textbackslash newrgbcolor\{userFillColour\}\{0 0 0\}\\
%\textbackslash psset\{linewidth=0.3,dotsep=1,hatchwidth=0.3,hatchsep=1.5,shadowsize=1,dimen=middle\}\\
%\textbackslash psset\{dotsize=0.7 2.5,dotscale=1 1,fillcolor=userFillColour\}\\
%\textbackslash psset\{arrowsize=1 2,arrowlength=1,arrowinset=0.25,tbarsize=0.7 5,bracketlength=0.15,rbracketlength=0.15\}\\
%\textbackslash makeatletter\textbackslash @ifundefined\{Pst@correctAnglefalse\}\{\}\{\textbackslash psset\{correctAngle=false\}\}\textbackslash makeatother\\
%\textbackslash begin\{pspicture\}(1.18,0)(21.18,19.02)\\
%\textbackslash pscustom[linecolor=red]\{\textbackslash psline(17.36,0)(11.18,19.02)\\
%\textbackslash psbezier(11.18,19.02)(11.18,19.02)(11.18,19.02)\\
%\textbackslash psline(11.18,19.02)(5,0)\\
%\textbackslash psline(5,0)(21.18,11.76)\\
%\textbackslash psline(21.18,11.76)(1.18,11.76)\\
%\textbackslash psline(1.18,11.76)(17.36,0)\\
%\textbackslash closepath\}\\
%\textbackslash rput(11.18,9.51)\{\textbackslash textcolor\{blue\}\{\textbackslash large 5\}\}\\
%\textbackslash end\{pspicture\}\\
%\end{tabular}
%</text>
%<text text-vert-align="top"
%	 text-icon="iconmode"
%	 anchor-point="(0.5,91)"
%	 text-hor-align="left"
%	 >
%\color{Brown}\tiny\begin{tabular}{@{}&gt;{\%}l@{}}
%&lt;?xml version="1.0" standalone="yes"?&gt;\\
%&lt;jpic x-min="1.18" x-max="21.18" y-min="0" y-max="19.02" auto-bounding="true"&gt;\\
%&lt;multicurve stroke-color="\#ff0000"\\
%	 points="(17.36,0);(17.36,0);(11.18,19.02);(11.18,19.02);(11.18,19.02);(11.18,19.02);\\
%	(11.18,19.02);(11.18,19.02);(5,0);(5,0);(5,0);(21.18,11.76);\\
%	(21.18,11.76);(21.18,11.76);(1.18,11.76);(1.18,11.76);(1.18,11.76);(17.36,0)"\\
%	 /&gt;\\
%&lt;text anchor-point="(11.18,9.51)"\\
%	 &gt;\\
%\textbackslash textcolor\{blue\}\{\textbackslash large 5\}\\
%&lt;/text&gt;\\
%&lt;/jpic&gt;
%\end{tabular}
%</text>
%<text text-vert-align="top"
%	 text-icon="iconmode"
%	 anchor-point="(0.5,117.45)"
%	 text-hor-align="left"
%	 >
%\color{Blue}\tiny\begin{tabular}{@{}l@{}}
%\textcolor{Brown}{\% Version: \^^dId: help.RedStar.jpe.pstricks,v 1.2 2013/02/28 21:37:47 vincentb1 Exp \^^d}\\
%\textbackslash ifx\textbackslash JPicIsIncluded\textbackslash undefined\\
%\textbackslash documentclass\{article\}\\
%\textbackslash usepackage[tightpage,active,psfixbb]\{preview\}\\
%\textbackslash usepackage\{pstricks\}\\
%\textbackslash begin\{document\}\\
%\textbackslash thispagestyle\{empty\}\\
%\textbackslash begin\{preview\}\\
%\textbackslash fi
%\end{tabular}
%</text>
%<text text-vert-align="top"
%	 text-icon="iconmode"
%	 anchor-point="(0.5,97.7)"
%	 text-hor-align="left"
%	 >
%\color{Brown}\tiny\begin{tabular}{@{}&gt;{\%\%}l@{}}
%Created by jPicEdt 1.6-pre1: mixed JPIC-XML/LaTeX format\\
%Sat Feb 23 13:55:10 CET 2013\\
%Begin JPIC-XML
%\end{tabular}
%</text>
%<text text-vert-align="top"
%	 text-icon="iconmode"
%	 anchor-point="(0.5,65)"
%	 text-hor-align="left"
%	 >
%\tiny\textcolor{Brown}{\%\%End JPIC-XML}
%</text>
%<text text-vert-align="top"
%	 text-icon="iconmode"
%	 anchor-point="(0.5,20)"
%	 text-hor-align="left"
%	 >
%\tiny\textcolor{Brown}{\%\%User Data}
%</text>
%<text text-vert-align="top"
%	 text-icon="iconmode"
%	 anchor-point="(0.5,17.5)"
%	 text-hor-align="left"
%	 >
%\color{Blue}\tiny\begin{tabular}{@{}l@{}}
%\textbackslash ifx\textbackslash JPicIsIncluded\textbackslash undefined\\
%\textbackslash end\{preview\}\\
%\textbackslash end\{document\}\\
%\textbackslash fi\\
%\textcolor{Brown}{\%\%\% Local Variables:}\\
%\textcolor{Brown}{\%\%\% mode: latex}\\
%\textcolor{Brown}{\%\%\% eval: (TeX-PDF-mode 0)}\\
%\textcolor{Brown}{\%\%\% End:}
%\end{tabular}
%</text>
%<g lift-pen="0"
%	 compound-mode="separate"
%	 closed="false"
%	>
%<multicurve points="(120,98);(121.25,98);(122.5,100.5);(122.5,101.75);(122.5,101.75);(122.5,106.75);
%	(122.5,106.75);(122.5,108);(123.75,108);(125,108);(123.75,108);(122.5,108);
%	(122.5,109.25);(122.5,109.25);(122.5,115.5);(122.5,115.5);(122.5,116.75);(121.25,118);
%	(120,118)"
%	 />
%</g>
%<g lift-pen="0"
%	 compound-mode="separate"
%	 closed="false"
%	>
%<multicurve points="(120,91.3);(121.25,91.3);(122.5,92.14);(122.5,92.56);(122.5,92.56);(122.5,94.23);
%	(122.5,94.23);(122.5,94.65);(123.75,94.65);(125,94.65);(123.75,94.65);(122.5,94.65);
%	(122.5,95.07);(122.5,95.07);(122.5,97.16);(122.5,97.16);(122.5,97.58);(121.25,98);
%	(120,98)"
%	 />
%</g>
%<g lift-pen="0"
%	 compound-mode="separate"
%	 closed="false"
%	>
%<multicurve points="(120,65.5);(121.25,65.5);(122.5,68.75);(122.5,70.38);(122.5,70.38);(122.5,76.88);
%	(122.5,76.88);(122.5,78.5);(123.75,78.5);(125,78.5);(123.75,78.5);(122.5,78.5);
%	(122.5,80.12);(122.5,80.12);(122.5,88.25);(122.5,88.25);(122.5,89.88);(121.25,91.5);
%	(120,91.5)"
%	 />
%</g>
%<g lift-pen="0"
%	 compound-mode="separate"
%	 closed="false"
%	>
%<multicurve points="(120,62.8);(121.25,62.8);(122.5,63.11);(122.5,63.28);(122.5,63.28);(122.5,63.9);
%	(122.5,63.9);(122.5,64.05);(123.75,64.05);(125,64.05);(123.75,64.05);(122.5,64.05);
%	(122.5,64.2);(122.5,64.2);(122.5,64.99);(122.5,64.99);(122.5,65.15);(121.25,65.3);
%	(120,65.3)"
%	 />
%</g>
%<g lift-pen="0"
%	 compound-mode="separate"
%	 closed="false"
%	>
%<multicurve points="(120,20.3);(121.25,20.3);(122.5,25.61);(122.5,28.27);(122.5,28.27);(122.5,38.89);
%	(122.5,38.89);(122.5,41.55);(123.75,41.55);(125,41.55);(123.75,41.55);(122.5,41.55);
%	(122.5,44.21);(122.5,44.21);(122.5,57.49);(122.5,57.49);(122.5,60.14);(121.25,62.8);
%	(120,62.8)"
%	 />
%</g>
%<g lift-pen="0"
%	 compound-mode="separate"
%	 closed="false"
%	>
%<multicurve points="(120,17.8);(121.25,17.8);(122.5,18.11);(122.5,18.27);(122.5,18.27);(122.5,18.89);
%	(122.5,18.89);(122.5,19.05);(123.75,19.05);(125,19.05);(123.75,19.05);(122.5,19.05);
%	(122.5,19.21);(122.5,19.21);(122.5,19.99);(122.5,19.99);(122.5,20.14);(121.25,20.3);
%	(120,20.3)"
%	 />
%</g>
%<g lift-pen="0"
%	 compound-mode="separate"
%	 closed="false"
%	>
%<multicurve points="(120,0);(121.25,0);(122.5,2.22);(122.5,3.34);(122.5,3.34);(122.5,7.79);
%	(122.5,7.79);(122.5,8.9);(123.75,8.9);(125,8.9);(123.75,8.9);(122.5,8.9);
%	(122.5,10.01);(122.5,10.01);(122.5,15.58);(122.5,15.58);(122.5,16.69);(121.25,17.8);
%	(120,17.8)"
%	 />
%</g>
%<text anchor-point="(125,108)"
%	 text-hor-align="left"
%	 >
%\parbox{50\unitlength}{\raggedright\large\IntlUserPreamble}
%</text>
%<text anchor-point="(125,94.65)"
%	 text-hor-align="left"
%	 >
%\parbox{50\unitlength}{\raggedright\large\IntlJpicXmlStart}
%</text>
%<text anchor-point="(125,78.5)"
%	 text-hor-align="left"
%	 >
%\parbox{50\unitlength}{\raggedright\large\IntlJpicXml}
%</text>
%<text anchor-point="(125,64.05)"
%	 text-hor-align="left"
%	 >
%\parbox{50\unitlength}{\raggedright\large\IntlJpicXmlEnd}
%</text>
%<text anchor-point="(125,41.55)"
%	 text-hor-align="left"
%	 >
%\parbox{50\unitlength}{\raggedright\large\IntlFormatted}
%</text>
%<text anchor-point="(125,19.05)"
%	 text-hor-align="left"
%	 >
%\parbox{50\unitlength}{\raggedright\large\IntlUserPostambleStart}
%</text>
%<text anchor-point="(125,8.9)"
%	 text-hor-align="left"
%	 >
%\parbox{50\unitlength}{\raggedright\large\IntlUserPostamble}
%</text>
%</jpic>
%%End JPIC-XML
%PSTricks content-type (pstricks.sty package needed)
%Add \usepackage{pstricks} in the preamble of your LaTeX file
%You can rescale the whole picture (to 80% for instance) by using the command \def\JPicScale{0.8}
\ifx\JPicScale\undefined\def\JPicScale{1}\fi
\psset{unit=\JPicScale mm}
\newrgbcolor{userFillColour}{0 0 0}
\psset{linewidth=0.3,dotsep=1,hatchwidth=0.3,hatchsep=1.5,shadowsize=1,dimen=middle}
\psset{dotsize=0.7 2.5,dotscale=1 1,fillcolor=userFillColour}
\psset{arrowsize=1 2,arrowlength=1,arrowinset=0.25,tbarsize=0.7 5,bracketlength=0.15,rbracketlength=0.15}
\makeatletter\@ifundefined{Pst@correctAnglefalse}{}{\psset{correctAngle=false}}\makeatother
\begin{pspicture}(0,0)(175,118)
\newrgbcolor{userFillColour}{0.69 0.93 0.93}
\pspolygon[linestyle=none,fillcolor=userFillColour,fillstyle=solid](0,91.3)(120,91.3)(120,65.3)(0,65.3)
\newrgbcolor{userFillColour}{0.56 0.93 0.56}
\pspolygon[linestyle=none,fillcolor=userFillColour,fillstyle=solid](0,62.8)(120,62.8)(120,20.3)(0,20.3)
\newrgbcolor{userFillColour}{0.86 0.86 0.86}
\pspolygon[linestyle=none,fillcolor=userFillColour,fillstyle=solid](0,17.8)(120,17.8)(120,0)(0,0)
\newrgbcolor{userFillColour}{0.86 0.86 0.86}
\pspolygon[linestyle=none,fillcolor=userFillColour,fillstyle=solid](0,118)(120,118)(120,98)(0,98)
\newrgbcolor{userFillColour}{1 0.71 0.76}
\pspolygon[linestyle=none,fillcolor=userFillColour,fillstyle=solid](0,98)(120,98)(120,91.3)(0,91.3)
\newrgbcolor{userFillColour}{1 0.71 0.76}
\pspolygon[linestyle=none,fillcolor=userFillColour,fillstyle=solid](0,65.3)(120,65.3)(120,62.8)(0,62.8)
\newrgbcolor{userFillColour}{1 0.71 0.76}
\pspolygon[linestyle=none,fillcolor=userFillColour,fillstyle=solid](0,20.3)(120,20.3)(120,17.8)(0,17.8)
\rput[tl](0.5,62.5){\color{Blue}\tiny\begin{tabular}{@{}l@{}}
\color{Brown}\%PSTricks content-type (pstricks.sty package needed)\\
\color{Brown}\%Add \textbackslash usepackage\{pstricks\} in the preamble of your LaTeX file\\
\color{Brown}\%You can rescale the whole picture (to 80\% for instance) by using the command \textbackslash def\textbackslash JPicScale\{0.8\}\\
\textbackslash ifx\textbackslash JPicScale\textbackslash undefined\textbackslash def\textbackslash JPicScale\{1\}\textbackslash fi\\
\textbackslash psset\{unit=\textbackslash JPicScale mm\}\\
\textbackslash newrgbcolor\{userFillColour\}\{0 0 0\}\\
\textbackslash psset\{linewidth=0.3,dotsep=1,hatchwidth=0.3,hatchsep=1.5,shadowsize=1,dimen=middle\}\\
\textbackslash psset\{dotsize=0.7 2.5,dotscale=1 1,fillcolor=userFillColour\}\\
\textbackslash psset\{arrowsize=1 2,arrowlength=1,arrowinset=0.25,tbarsize=0.7 5,bracketlength=0.15,rbracketlength=0.15\}\\
\textbackslash makeatletter\textbackslash @ifundefined\{Pst@correctAnglefalse\}\{\}\{\textbackslash psset\{correctAngle=false\}\}\textbackslash makeatother\\
\textbackslash begin\{pspicture\}(1.18,0)(21.18,19.02)\\
\textbackslash pscustom[linecolor=red]\{\textbackslash psline(17.36,0)(11.18,19.02)\\
\textbackslash psbezier(11.18,19.02)(11.18,19.02)(11.18,19.02)\\
\textbackslash psline(11.18,19.02)(5,0)\\
\textbackslash psline(5,0)(21.18,11.76)\\
\textbackslash psline(21.18,11.76)(1.18,11.76)\\
\textbackslash psline(1.18,11.76)(17.36,0)\\
\textbackslash closepath\}\\
\textbackslash rput(11.18,9.51)\{\textbackslash textcolor\{blue\}\{\textbackslash large 5\}\}\\
\textbackslash end\{pspicture\}\\
\end{tabular}}
\rput[tl](0.5,91){\color{Brown}\tiny\begin{tabular}{@{}>{\%}l@{}}
<?xml version="1.0" standalone="yes"?>\\
<jpic x-min="1.18" x-max="21.18" y-min="0" y-max="19.02" auto-bounding="true">\\
<multicurve stroke-color="\#ff0000"\\
	 points="(17.36,0);(17.36,0);(11.18,19.02);(11.18,19.02);(11.18,19.02);(11.18,19.02);\\
	(11.18,19.02);(11.18,19.02);(5,0);(5,0);(5,0);(21.18,11.76);\\
	(21.18,11.76);(21.18,11.76);(1.18,11.76);(1.18,11.76);(1.18,11.76);(17.36,0)"\\
	 />\\
<text anchor-point="(11.18,9.51)"\\
	 >\\
\textbackslash textcolor\{blue\}\{\textbackslash large 5\}\\
</text>\\
</jpic>
\end{tabular}}
\rput[tl](0.5,117.45){\color{Blue}\tiny\begin{tabular}{@{}l@{}}
\textcolor{Brown}{\% Version: \^^dId: help.RedStar.jpe.pstricks,v 1.2 2013/02/28 21:37:47 vincentb1 Exp \^^d}\\
\textbackslash ifx\textbackslash JPicIsIncluded\textbackslash undefined\\
\textbackslash documentclass\{article\}\\
\textbackslash usepackage[tightpage,active,psfixbb]\{preview\}\\
\textbackslash usepackage\{pstricks\}\\
\textbackslash begin\{document\}\\
\textbackslash thispagestyle\{empty\}\\
\textbackslash begin\{preview\}\\
\textbackslash fi
\end{tabular}}
\rput[tl](0.5,97.7){\color{Brown}\tiny\begin{tabular}{@{}>{\%\%}l@{}}
Created by jPicEdt 1.6-pre1: mixed JPIC-XML/LaTeX format\\
Sat Feb 23 13:55:10 CET 2013\\
Begin JPIC-XML
\end{tabular}}
\rput[tl](0.5,65){\tiny\textcolor{Brown}{\%\%End JPIC-XML}}
\rput[tl](0.5,20){\tiny\textcolor{Brown}{\%\%User Data}}
\rput[tl](0.5,17.5){\color{Blue}\tiny\begin{tabular}{@{}l@{}}
\textbackslash ifx\textbackslash JPicIsIncluded\textbackslash undefined\\
\textbackslash end\{preview\}\\
\textbackslash end\{document\}\\
\textbackslash fi\\
\textcolor{Brown}{\%\%\% Local Variables:}\\
\textcolor{Brown}{\%\%\% mode: latex}\\
\textcolor{Brown}{\%\%\% eval: (TeX-PDF-mode 0)}\\
\textcolor{Brown}{\%\%\% End:}
\end{tabular}}
{%
\pscustom[]{\psbezier(120,98)(121.25,98)(122.5,100.5)(122.5,101.75)
\psline(122.5,101.75)(122.5,106.75)
\psbezier(122.5,108)(123.75,108)(125,108)
\psbezier(123.75,108)(122.5,108)(122.5,109.25)
\psline(122.5,109.25)(122.5,115.5)
\psbezier(122.5,116.75)(121.25,118)(120,118)
}
}%
{%
\pscustom[]{\psbezier(120,91.3)(121.25,91.3)(122.5,92.14)(122.5,92.56)
\psline(122.5,92.56)(122.5,94.23)
\psbezier(122.5,94.65)(123.75,94.65)(125,94.65)
\psbezier(123.75,94.65)(122.5,94.65)(122.5,95.07)
\psline(122.5,95.07)(122.5,97.16)
\psbezier(122.5,97.58)(121.25,98)(120,98)
}
}%
{%
\pscustom[]{\psbezier(120,65.5)(121.25,65.5)(122.5,68.75)(122.5,70.38)
\psline(122.5,70.38)(122.5,76.88)
\psbezier(122.5,78.5)(123.75,78.5)(125,78.5)
\psbezier(123.75,78.5)(122.5,78.5)(122.5,80.12)
\psline(122.5,80.12)(122.5,88.25)
\psbezier(122.5,89.88)(121.25,91.5)(120,91.5)
}
}%
{%
\pscustom[]{\psbezier(120,62.8)(121.25,62.8)(122.5,63.11)(122.5,63.28)
\psline(122.5,63.28)(122.5,63.9)
\psbezier(122.5,64.05)(123.75,64.05)(125,64.05)
\psbezier(123.75,64.05)(122.5,64.05)(122.5,64.2)
\psline(122.5,64.2)(122.5,64.99)
\psbezier(122.5,65.15)(121.25,65.3)(120,65.3)
}
}%
{%
\pscustom[]{\psbezier(120,20.3)(121.25,20.3)(122.5,25.61)(122.5,28.27)
\psline(122.5,28.27)(122.5,38.89)
\psbezier(122.5,41.55)(123.75,41.55)(125,41.55)
\psbezier(123.75,41.55)(122.5,41.55)(122.5,44.21)
\psline(122.5,44.21)(122.5,57.49)
\psbezier(122.5,60.14)(121.25,62.8)(120,62.8)
}
}%
{%
\pscustom[]{\psbezier(120,17.8)(121.25,17.8)(122.5,18.11)(122.5,18.27)
\psline(122.5,18.27)(122.5,18.89)
\psbezier(122.5,19.05)(123.75,19.05)(125,19.05)
\psbezier(123.75,19.05)(122.5,19.05)(122.5,19.21)
\psline(122.5,19.21)(122.5,19.99)
\psbezier(122.5,20.14)(121.25,20.3)(120,20.3)
}
}%
{%
\pscustom[]{\psbezier(120,0)(121.25,0)(122.5,2.22)(122.5,3.34)
\psline(122.5,3.34)(122.5,7.79)
\psbezier(122.5,8.9)(123.75,8.9)(125,8.9)
\psbezier(123.75,8.9)(122.5,8.9)(122.5,10.01)
\psline(122.5,10.01)(122.5,15.58)
\psbezier(122.5,16.69)(121.25,17.8)(120,17.8)
}
}%
\rput[l](125,108){\parbox{50\unitlength}{\raggedright\large\IntlUserPreamble}}
\rput[l](125,94.65){\parbox{50\unitlength}{\raggedright\large\IntlJpicXmlStart}}
\rput[l](125,78.5){\parbox{50\unitlength}{\raggedright\large\IntlJpicXml}}
\rput[l](125,64.05){\parbox{50\unitlength}{\raggedright\large\IntlJpicXmlEnd}}
\rput[l](125,41.55){\parbox{50\unitlength}{\raggedright\large\IntlFormatted}}
\rput[l](125,19.05){\parbox{50\unitlength}{\raggedright\large\IntlUserPostambleStart}}
\rput[l](125,8.9){\parbox{50\unitlength}{\raggedright\large\IntlUserPostamble}}
\end{pspicture}
%%User Data
\ifx\JPicIsIncluded\undefined
\end{preview}
\end{document}
\fi
%%% Local Variables:
%%% mode: latex
%%% eval: (TeX-PDF-mode 0)
%%% End:
